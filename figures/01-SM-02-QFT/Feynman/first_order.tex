\begin{figure}[ht]
	\centering
	\begin{tikzpicture}
		\begin{feynman}
			\vertex (a) {\(\phi\)};
			\vertex [right=of a] (b);
			\vertex [above right=of b] (f1) {\(\psi\)};
			\vertex [below right=of b] (f2) {\(\psi^\dagger\)};
			\diagram* {
				(a) -- [scalar, momentum'={\footnotesize\(p\)}] (b),
				(b) -- [fermion, momentum={[arrow shorten=0.25]\footnotesize\(q_1\)}] (f1),
				(b) -- [anti fermion, momentum={[arrow shorten=0.25]\footnotesize\(q_2\)}] (f2),
			};
		\end{feynman}
	\end{tikzpicture}
	\hspace{2cm}
	% nucleon-anti nucleon annihilation
	\begin{tikzpicture}
		\begin{feynman}
			\vertex (a);
			\vertex [above left=of a] (i1) {\(\psi\)};
			\vertex [right=of a] (b) {\(\phi\)};
			\vertex [below left=of a] (i2) {\(\psi^\dagger\)};
			\diagram* {
				(i1) -- [fermion, momentum'={[arrow shorten=0.25]\footnotesize\(q_1\)}] (a),
				(i2) -- [anti fermion, momentum'={[arrow shorten=0.25]\footnotesize\(q_2\)}] (a),
				(a) -- [scalar, momentum={\footnotesize\(p\)}] (b),
			};
		\end{feynman}
	\end{tikzpicture}
	\vspace{5mm}
	\caption{Feynman diagrams for meson decay (left) and nucleon-antinucleon annihilation (right).}
	\label{fig:01_qft_interactions_feynman_first_order}
\end{figure}