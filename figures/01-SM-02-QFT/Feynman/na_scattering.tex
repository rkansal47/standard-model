\begin{figure}[ht]
	\centering
	\captionsetup{justification=centering}
	\begin{tikzpicture}
		\begin{feynman}
			\vertex (a);
			\vertex [below=of a] (b);
			\vertex [above left=of a] (i1);
			\vertex [below left=of b] (i2);
			\vertex [above right=of a] (f1);
			\vertex [below right=of b] (f2);
			\diagram* {
				(a) -- [scalar, edge label={\footnotesize$k$}] (b),
				(i1) -- [fermion, edge label'={\footnotesize$q_{i1}$}] (a),
				(i2) -- [anti fermion, edge label={\footnotesize$q_{i2}$}] (b),
				(a) -- [fermion, edge label'={\footnotesize$q_{f1}$}] (f1),
				(b) -- [anti fermion, edge label={\footnotesize$q_{f2}$}] (f2),
			};
		\end{feynman}
	\end{tikzpicture}
	\hspace{3cm}
	% s-channel
	\raisebox{7mm}{
	\begin{tikzpicture}
		\begin{feynman}
			\vertex (a);
			\vertex [right=of a] (b);
			\vertex [above left=of a] (i1);
			\vertex [below left=of a] (i2);
			\vertex [above right=of b] (f1);
			\vertex [below right=of b] (f2);
			\diagram* {
				(a) -- [scalar, edge label'={\footnotesize$k$}] (b),
				(i1) -- [fermion, edge label={\footnotesize$q_{i1}$}] (a),
				(i2) -- [anti fermion, edge label'={\footnotesize$q_{i2}$}] (a),
				(b) -- [anti fermion, edge label={\footnotesize$q_{f1}$}] (f1),
				(b) -- [fermion, edge label'={\footnotesize$q_{f2}$}] (f2),
			};
		\end{feynman}
	\end{tikzpicture}
	}
	\vspace{5mm}
	\caption{The two lowest order nucleon-antinucleon scattering diagrams.}
	\label{fig:01_qft_interactions_feynman_na_scattering}
\end{figure}