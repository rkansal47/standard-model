
\chapter{Symmetries in physics}
\label{sec:01_symmetries}

\begin{center}
	\centering
	\noindent
	\textit{Perfectly balanced, as all things should be.} --- Thanos
\end{center}

Symmetry is a powerful and beautiful way to understand nature. 
Intuitively, a symmetry is a transformation that leaves an object unchanged.
For example, a plain square has a four-fold rotational symmetry: it looks identical rotated once, twice, thrice, or four times by $90^{\circ}$.

Similarly, in physics, a symmetry is a transformation that leaves the laws of physics unchanged.
Electromagnetism, for example, is invariant to translations in space or time: electric charges and currents should behave the same in San Diego 5 years ago as in Geneva today.
Understanding such symmetries, and accounting for them in our mathematical formulation, has been a guiding principle in the development of the SM over the 20th century, and is one in understanding it as well.

In recent years, symmetries have also guided the development of machine learning algorithms in becoming more powerful and efficient.
A particular focus is placed in this dissertation on such \textit{equivariant} algorithms, which respect the symmetries and 
\textit{inductive biases} of our high energy physics data.
This chapter lays the foundation for these ideas, which we discuss in more detail in Chapter~\ref{sec:03_ml} and contribute to in Chapter~\ref{sec:06_lgae}.

In this chapter, we first introduce the framework for describing symmetries, group theory, in Section~\ref{sec:01_symmetries_gt}.
We then describe Lie algebras for continuous symmetries, and derive representations for the algebra corresponding to 3D rotations, in Section~\ref{sec:01_symmetries_so3}.
% in Sections~\ref{sec:01_symmetries_continuous} and~\ref{sec:01_symmetries_lie}, respectively.
% We then derive representations of the Lie algebra corresponding to 3D rotations in Section~\ref{sec:01_symmetries_so3}.
We conclude in Section~\ref{sec:01_symmetries_poincare} with a discussion of the Lorentz and Poincaré groups, comprising the fundamental symmetries of spacetime, whose irreducible representations are what we call particles.
% This  lays the foundations for describing \textit{group-equivariant} machine learning algorithms, an exciting area of research in AI that will be a focus of Chapter~\ref{sec:06_ml4jets}.

\section{Group theory}
\label{sec:01_symmetries_gt}

The mathematical formalism for describing symmetries is called \textit{group theory}.

\begin{definition}
\label{def:01_group}
The fundamental object in group theory is a \textit{group}, defined as a pair $(G, \bullet)$, where $G$ is a set and $\bullet: G \times G \rightarrow G$ is the group operation, which together satisfies the following properties:
\begin{enumerate}[i)]
	\item Associativity: $\forall a, b, c \in G: (a \bullet b) \bullet c = a \bullet (b \bullet c)$.
	\item Identity element: $\exists e \in G: \forall a \in G: a \bullet e = e \bullet a = a$.
	\item Inverse element: $\forall a \in G: \exists a^{-1} \in G: a \bullet a^{-1} = a^{-1} \bullet a = e$.
\end{enumerate}
\end{definition}

\begin{definition}
\label{def:01_abelian}
Note from Definition~\ref{def:01_group} that the group operation is not necessarily commutative ($\forall a, b \in G: a \bullet b = b \bullet a$).
If this condition does hold, the group is called an \textit{abelian} group.
\end{definition}

\begin{example}
\label{example:01_square}
To formalize the four-fold rotation symmetry of a square discussed above, we can define the group $\mathbb Z_4$  as ($\{0, 1, 2, 3\}$, $+_4$), where $+_4$ is addition modulo 4, and the elements of the set can represent rotations by $0^{\circ}$, $90^{\circ}$, $180^{\circ}$, and $270^{\circ}$, respectively.
One can check that $\mathbb Z_4$ satisfies all the properties of an abelian group.
\end{example}

\subsubsection{Group representations}

To make the abstract mathematical structure of the group more concrete, we next consider \textit{representations} of groups.

\begin{definition}
\label{def:01_representation}
A group representation $R$, of dimension $d$, is a mapping of the group elements to $d\times d$ matrices $D(g)$ in some $d$-dimensional vector space $V$, such that the group operation is preserved: $D(g_1)D(g_2) = D(g_1 \bullet g_2)$.
Necessarily, this means that $D(e) = \identity$, the identity matrix of $V$.
Representations of a group are not unique, and arbitrarily many new represenations can be constructed simply by taking tensor sums and products, denoted by the $\oplus$ and $\otimes$ symbols respectively, of existing ones.
\end{definition}

\begin{definition}
\label{def:01_irreps}
An \textit{irreducible representation} (irrep) is one with no non-trivial invariant subspaces, i.e., it cannot be decomposed into the tensor sums of smaller-dimensional representations.\footnote{Technically, certain pathological reducible representations of non-compact groups also cannot be decomposed into irreps, so ``non-decomposability'' is a necessary but insufficient condition for irreps.}
\end{definition}

\begin{example}
\label{example:01_square_representation}
The group $\mathbb Z_4$ from Example~\ref{example:01_square} can be represented simply as scalar complex numbers ($V = \mathbb C$):
\begin{equation}
	\label{eq:01_z4_representation}
	\begin{array}{cccc}
	0 & 1 & 2 & 3 \\
	\downarrow & \downarrow & \downarrow & \downarrow \\
	1 & e^{i\frac{\pi}{2}} & e^{i\pi} & e^{i\frac{3\pi}{2}}
	\end{array}
\end{equation}
One can check this satisfies the conditions of Definition~\ref{def:01_representation}, and since it is 1-dimensional, it is also irreducible.
\end{example}

\begin{definition}
\label{def:01_symmetry_regular_representation}
Every group has a $|G|$-dimensional \textit{regular representation} $R^{\mathrm{reg}}$, where $|G|$ is the number elements of the group, called the \textit{order} of the group.
The vector space $V = \mathrm{span}\{\ket{g}|g\in G\}$, and the representation is defined such that
\begin{equation}
	\label{eq:01_regular_representation}
	D^{\mathrm{reg}}(g)\ket{h} = \ket{gh}.
\end{equation}
\end{definition}

\begin{example}
\label{example:01_z4_regular}
For our $\mathbb Z_4$ group, we can use the set of four basis vectors $\{\ket{0} = \mathbf{e}_0, \ket{1} = \mathbf{e}_1, \ket{2} = \mathbf{e}_2, \ket{3} = \mathbf{e}_3\}$ in $\mathbb R^4$, and derive the matrices $D^{\mathrm{reg}}(g)$ such that they transform $\ket{g}$ according to the respective group operations:
\setlength{\arraycolsep}{10pt}
\begin{equation} \begin{split}
	\label{eq:01_z4_regular}
	D^{\mathrm{reg}}(0) &= \begin{pmatrix}
		1 & 0 & 0 & 0 \\
		0 & 1 & 0 & 0 \\
		0 & 0 & 1 & 0 \\
		0 & 0 & 0 & 1
	\end{pmatrix}, \quad
	D^{\mathrm{reg}}(1) = \begin{pmatrix}
		0 & 1 & 0 & 0 \\
		0 & 0 & 1 & 0 \\
		0 & 0 & 0 & 1 \\
		1 & 0 & 0 & 0
	\end{pmatrix}, \quad \\[1em]
	D^{\mathrm{reg}}(2) &= \begin{pmatrix}
		0 & 0 & 1 & 0 \\
		0 & 0 & 0 & 1 \\
		1 & 0 & 0 & 0 \\
		0 & 1 & 0 & 0
	\end{pmatrix}, \quad
	D^{\mathrm{reg}}(3) = \begin{pmatrix}
		0 & 0 & 0 & 1 \\
		1 & 0 & 0 & 0 \\
		0 & 1 & 0 & 0 \\
		0 & 0 & 1 & 0
	\end{pmatrix}.
\end{split} \end{equation}
\end{example}

The regular representation has some fun properties, such as its reducibility into irreps with each irrep appearing as many times in the decomposition as its dimension.
For us, it will mostly serve as a useful way to think about the \textit{adjoint} representation we will encounter below.

\subsubsection{Continuous symmetries}
\label{sec:01_symmetries_continuous}

Symmetries can be \textit{discrete}, as above, as well as continuous.

\begin{example}
\label{example:01_circle}
A circle has a continuous 2D rotational symmetry; rotations by any angle $\theta$ leave it invariant.
This corresponds to the \textit{special orthogonal} group in 2-dimensions $\SO[2]$.
\end{example}

\begin{definition}
\label{def:01_son}
More generally, the orthogonal group in $n$ dimensions, $\OO[n]$, is defined as	the group of orthogonal, or ``distance-preserving'', $n \times n$ matrices $M$, s.t. $MM^T = \identity$. 
The \textit{special orthogonal} group $\SO[n]$ is the subgroup of $n \times n$ orthogonal matrices with determinant 1, essentially retaining only rotations while removing reflections.
\end{definition}

As their definition suggests, the \SO[n] group elements have a natural representation as the $n \times n$ rotation matrices.
For \SO[2], these are of the form:
\begin{equation}
	\label{eq:01_so2}
	M(\theta) = \begin{pmatrix}
		\cos \theta & -\sin \theta \\
		\sin \theta & \cos \theta
	\end{pmatrix},
\end{equation}
where $\theta \in [0, 2\pi)$ is the angle of rotation.
These $n \times n$ matrix representations are called the \textit{fundamental} or \textit{defining} representations of \SO[n].

\begin{definition}
\label{def:01_sun}
\SO[2] is \textit{isomorphic} --- meaning identical to in terms of its group-theoretic properties --- to the \textit{unitary} group \UU[1].
The \textit{unitary} group \UU[n] is the group of $n \times n$ unitary matrices, i.e., those satisfying $M^\dagger M = MM^\dagger = \identity$, where $M^\dagger$ is the conjugate transpose, or Hermitian conjugate (h.c.) of $M$.
The special unitary group \SU[n], again is the subgroup of $n \times n$ unitary matrices with determinant 1.
As we will soon see, these groups effectively define the structure of the SM.
\end{definition}

\UU[1] has the simple 1D fundamental representation:
\begin{equation}
	\label{eq:01_u1}
	M(\theta) = e^{i\theta},
\end{equation}
i.e., all complex numbers of unit magnitude.


\begin{definition}
\label{def:01_compact}
An infinite group is \textit{compact} if a group-invariant sum or integral over the group elements is finite.
\UU[1] is compact, as
\begin{equation}
	\label{eq:01_u1_integral}
	\int_0^{2\pi} d\theta = 2\pi
\end{equation}
is finite. 
Indeed, all \SO[n] and \SU[n] groups are compact.
\end{definition}

Examples of important non-compact groups include the group of translations in $n$ dimensions and the Lorentz group, which we will discuss in detail in Section~\ref{sec:01_symmetries_poincare}.

\section{Lie algebras}
\label{sec:01_symmetries_lie}

We next introduce the concepts of Lie groups and Lie algebras, which are highly useful in understanding the structure and representations of continuous groups.

\begin{definition}
\label{def:01_lie_group}
A \textit{Lie group} is a group that is also a differentiable manifold, or ``smooth''.
Virtually all continuous groups we consider in physics are Lie groups.
What this means is that we can think of the operation of any arbitrary group element as equivalent to $N$ successive infinitesimal operations of the form 
\begin{equation}
	\label{eq:01_lie_group_infinitesimal}
	g(\varepsilon_A) = \identity + i \varepsilon_A T_A,
\end{equation}
where $\varepsilon_A$ are infinitesimal and indexing the continuous group parameters, e.g. rotation angles for \SO[n], $T_A$ are called the \textit{generators} of the group,
% \footnote{They are named so because they \textit{generate} the transformations of the group; for example, we will see that the generators for rotations correspond to angular momentum, which is what is required for a body to rotate.} 
and we are using Einstein notation, implicitly summing over the index $A$.
Thus, for a general element $g(\theta_A)$, where $\theta_A = N \varepsilon_A$ as defined above, we have
\begin{equation}
	\label{eq:01_lie_group_finite}
	g(\theta_A) = \left(\identity + i \frac{\theta_A}{N} T_A\right)^N \xrightarrow{N \rightarrow \infty} e^{i\theta_A T_A}.
\end{equation}
This is somewhat analogous to Taylor expansion in calculus, except for Lie groups only the first order / derivative term is necessary to capture the group behavior.\footnote{This is because, based on the Campbell-Baker-Hausdorff~\cite{enwiki:1183926638} formula, higher order terms in the expansion of exponential form of $g$ in Eq.~\ref{eq:01_lie_group_finite} involve only commutators of the generators.}
% More importantly, the generators $T_A$ are highly useful in understanding the structure and representations of the Lie group.
\end{definition}

\begin{definition}
\label{def:01_lie_algebra}
The \textit{Lie algebra}, $\mathfrak{g}$, of a group is defined by the set of commutation relations between its generators:\footnote{An \textit{algebra} $(V, \bullet)$ is a vector space $V$ with a bilinear operation $\bullet: V \times V \rightarrow V$.
Examples include the cross product of vectors and matrix multiplication of square matrices.
The Lie algebra is the special case where $\bullet$ is the commutator.}
\begin{equation}
	\label{eq:01_lie_algebra}
	[T_A, T_B] = i f_{ABC} T_C,
\end{equation}
where $[T_A, T_B] = T_A T_B - T_B T_A$ is the commutator of $T_A$ and $T_B$, and $f_{ABC}$ are called the \textit{structure constants} of $\mathfrak{g}$.
As $[T_A, T_B] = -[T_B, T_A]$, the structure constants must be totally antisymmetric in the swapping of their indices.
% Commutators also carry the following useful property, known as the Jacobi identity:
% \begin{equation}
% 	\label{eq:01_jacobi}
% 	[T_A, [T_B, T_C]] + [T_B, [T_C, T_A]] + [T_C, [T_A, T_B]] = 0.
% \end{equation}
\end{definition}

\begin{example}
\label{example:01_u1_lie}
For the \UU[1] group,  we can see directly from Eq.~\ref{eq:01_u1} that the sole generator of the group is $T = 1$.
This has the rather uninteresting Lie algebra $\mathfrak{u}(1)$ of $[1, 1] = 0$, stemming from the fact that the group is abelian.
Next, we look at the more interesting \SO[3] and \SU[2] groups, where the power of Lie algebras shines.
\end{example}


% \subsection{Representations of the \texorpdfstring{\so[3] and \su[2]}{so(3) and su(2)} algebras}
\label{sec:01_symmetries_so3}

\subsubsection{Fundamental and adjoint representations of the \so[3] and \su[2] algebras}

We now turn to the derivation of the Lie algebras and representations of the \SO[3] and \SU[2] groups, both because of their importance in physics, and because the derivation introduces a number of useful concepts and definitions for the following sections.
\SO[3] and \SU[2] are very closely related: \SU[2] is a \textit{double cover} of \SO[3], which means that every rotation in \SO[3] can be mapped to two elements of \SU[2].
Importantly, however, they are locally isomorphic near the identity, meaning they have the same Lie algebra.

We can derive the generators of \SO[3] by using the properties of the special orthogonal group ($R^TR=\identity$).
From Eq.~\ref{eq:01_lie_group_infinitesimal}, we have
\begin{equation} 
	\begin{split}
	\label{eq:01_so3_generators_derivation}
	R(\varepsilon) &\approxeq \identity + i \varepsilon T \\
	R^TR &= \identity + i \varepsilon (T^T + T) + \mathcal O(\varepsilon^2) \mustequal \identity \\
	\Rightarrow T^T &= -T.
\end{split} 
\end{equation}
Thus $T$ are antisymmetric matrices, of which for $N = 3$ dimensions there are three linearly independent ones:
\begin{equation}
	\label{eq:01_so3_generators}
	J_x = i\begin{pmatrix}
		0 & 0 & 0 \\
		0 & 0 & -1 \\
		0 & 1 & 0
	\end{pmatrix}, \quad
	J_y = i\begin{pmatrix}
		0 & 0 & 1 \\
		0 & 0 & 0 \\
		-1 & 0 & 0
	\end{pmatrix}, \quad
	J_z = i\begin{pmatrix}
		0 & -1 & 0 \\
		1 & 0 & 0 \\
		0 & 0 & 0
	\end{pmatrix},
\end{equation}
labeled as $x$, $y$, $z$ as they represent rotations around the respective axes.
The factor of $i$ ensures the reality of the infinitesimal rotations in Eq.~\ref{eq:01_so3_generators_derivation} and also that the generators are Hermitian.\footnote{Note that the conventions around this factor are inconsistent in the literature and likely, despite our best efforts, will be inconsistent in this chapter as well.}
These provide us with the fundamental representation of \so[3], and should be familiar as the angular momentum operators in quantum mechanics (QM).
By exponentiating these, as in Eq.~\ref{eq:01_lie_group_finite}, we obtain the fundamental representation of the \SO[3] group: $R(\vec{\theta}) = e^{i\theta_i J_i}$.
% Note that conventions around the factor of $i$ in the representations and commutation relations are annoyingly inconsistent in the literature and likely, despite our best efforts, in this chapter as well.

To find the fundamental representation of \su[2], we can follow the same procedure as above, using the unitarity constraint $R^\dagger R = \identity$ for $N = 2$ dimensional complex matrices, which yields:
\begin{equation} 
	\label{eq:01_su2_generators}
	\begin{split}
	\setlength{\arraycolsep}{8pt}
	T_1 = \frac{1}{2}\sigma_x = \frac{1}{2}\begin{pmatrix}
		0 & 1 \\
		1 & 0
	\end{pmatrix}, \quad
	&T_2 = \frac{1}{2}\sigma_y = \frac{1}{2}\begin{pmatrix}
		0 & -i \\
		i & 0
	\end{pmatrix}, \\[1em]
	T_3 = \frac{1}{2}\sigma_z = &\frac{1}{2}\begin{pmatrix}
		1 & 0 \\
		0 & -1
	\end{pmatrix},
\end{split} 
\end{equation}
where $\sigma_i$ are the Pauli matrices --- the angular momentum operators for the spin of spin-$1/2$ particles in QM.
Either set of generators yield the following Lie algebra of both groups:
\begin{equation}
	\label{eq:01_so3_su2_lie_algebra}
	[T_A, T_B] = i \epsilon_{ABC} T_C,
\end{equation}
where the structure constants $f_{ABC}$ of the algebra are simply $\epsilon_{ABC}$, the totally antisymmetric Levi-Civita tensor.

Structure constants themselves furnish the following representation of the corresponding Lie algebra:
\begin{equation}
	\label{eq:01_adjoint}
	[T_A]_{BC} = -i f_{ABC}.
\end{equation}
This can be confirmed by plugging this representation into the commutator in Eq.~\ref{eq:01_lie_algebra} and using the Jacobi identity~\cite{Jacobi1862}.
As $B, C$ index the number of generators, we see that this representation has a dimension equal to the number of generators of the Lie algebra, and it is called its \textit{adjoint} representation.
It is analogous to the regular representation (Definition~\ref{def:01_symmetry_regular_representation}) for a Lie algebra, with the underlying vector space spanned by the generators $V = \mathrm{span}\{\ket{T_A}\}$ and the requirement that $D(T_A)\ket{T_B} = i f_{ABC} \ket{T_C}$.

\begin{definition}
\label{def:01_lie_group_dim}
The dimension of a Lie group is defined as the number of generators of the group.
Thus, it is the same as the dimension of the adjoint representation.
\end{definition}

As it turns out, for \so[3] and \su[2], the adjoint representation $[T_A]_{BC} = -i \varepsilon_{ABC}$ is simply the fundamental representation of \so[3].
More generally, the dimensions of the fundamental and adjoint representations of \SO[n] and \SU[n] are given in Tab.~\ref{tab:01_so_su_dimensions}.	
The significance of these representations, as we will see, is that the force carriers (i.e., gauge bosons) of the SM live in the adjoint representation of their associated gauge group, while the matter particles live in either their fundamental or trivial representations.

\begin{table}[ht!]
	\centering
	\renewcommand{\arraystretch}{1.5}
	\setlength{\tabcolsep}{10pt}
	\begin{tabular}{c|c|c}
		\textbf{Group} & \textbf{dim(Fundamental)} & \textbf{dim(Adjoint)} \\
		\hline
		% \SO[2] & 1 & 1 \\
		% \SO[3] & 3 & 3 \\
		\SO[n] & $n$ & $n(n-1)/2$ \\
		% \SU[2] & 2 & 3 \\
		% \SU[3] & 3 & 8 \\
		\SU[n] & $n$ & $n^2 - 1$ \\
	\end{tabular}
	\vspace{1em}
	\caption{Dimensions of the fundamental and adjoint representations of the \SO[n] and \SU[n] groups.}
	\label{tab:01_so_su_dimensions}
\end{table}

\subsubsection{General representations}

So far we have discussed two representations of the \so[3] and \su[2] algebras.
The general representations can be derived in much the same way as finding the eigenstates of the angular momentum operator in QM.
We first choose a basis in which one of the generators, conventionally $J_z$, is diagonal, and label eigenvectors of $J_z$ as $\ket{m}$ with eigenvalue $m$:
\begin{equation}
	\label{eq:01_so3_su2_eigenstates}
	J_z\ket{m} = m\ket{m}.
\end{equation}
These eigenvectors, by definition, form a basis for the representations of the generators, so counting them tells us the dimensions of allowed representation.
To do so, we define the ``raising'' and ``lowering'' operators $J_{\pm} = J_x \pm i J_y$, with commutation relations
\begin{equation}
	\label{eq:01_so3_su2_raise_lower}
	[J_z, J_{\pm}] = \pm J_{\pm}, \quad [J_+, J_-] = 2J_z.
\end{equation}
These are named so because
\begin{equation}
	\label{eq:01_so3_su2_raise_lower_eigenstates}
	J_zJ_\pm\ket{m} = [J_\pm Jz \pm J_\pm]\ket{m} = (m \pm 1)J_\pm\ket{m},
\end{equation}										
i.e., $J_\pm\ket{m}$ are eigenvectors of $J_z$ with eigenvalues $m \pm 1$, implying
\begin{equation}
	\label{eq:01_so3_su2_raise_lower_operation}
	J_\pm\ket{m} = c^\pm_{m\pm1}\ket{m \pm 1},
\end{equation}
where $c_{m\pm1}$ are normalization constants.
Now if we assume that the representation is finite-dimensional and label the highest-weight state $\ket{j}$ --- such that $J_+\ket{j} = 0 \Leftrightarrow c^+_{j+1} = 0$ --- we can iteratively lower the state and solve for the normalization constants until we reach the lowest-weight state.
By doing so we find that $c^-_{-j-1} = 0 \Rightarrow$ the lowest weight state is in fact $\ket{-j}$.\footnote{See, for example, Chapter IV.2 in Zee~\cite{Zee:2016fuk} for a more detailed derivation.}

Thus, we conclude the algebra allows $2j+1$-dimensional representations spanned by $\{\ket{-j}, \ket{-j+1}, \ldots, \ket{j-1}, \ket{j}\}$, with $j \in \mathbb Z^{\geq0}/2$ (non-negative integers and half-integers only).
Each possible $j$ indexes a different representation of the group, and any eigenstate can thus be labeled by $\ket{j, m}$.
We have already seen the $j = 1/2$ and $j = 1$ representations explicitly in Eqs.~\ref{eq:01_su2_generators} and \ref{eq:01_so3_generators}, respectively, while the $j = 0$ is simply the trivial representation of the group ($D(g) = 1 \,\ \forall g \in G$).

\begin{definition}
\label{def:01_casimir}
More generally, irreducible representations of a group are labeled by eigenvalues of the \textit{Casimir invariants}, or Casimirs, of the group.
Casimirs are operators that commute with all generators of the group.
For \so[3] and \su[2], there is only one Casimir,
\begin{equation}
\label{eq:01_so3_su2_casimir}
J^2 = J_x^2 + J_y^2 + J_z^2.
\end{equation}
This is the total angular momentum operator, which we know from QM commutes with all the $J_i$s and for any eigenstate $\ket{j, m}$ has eigenvalue $j(j+1)$:
\begin{equation}
\label{eq:01_so3_su2_casimir_eigenvalue}
J^2\ket{j, m} = j(j+1)\ket{j, m}.
\end{equation}
As expected, since the Casimir commutes with all the generators, its eigenvalues depend only on the irrep $j$.
We have also seen that individual states can be further labeled using the eigenvalues of a set of maximally commuting operators, in this case $\{J^2, J_z\}$.
\end{definition}

These representations can directly be used to derive those corresponding to the \SU[2] and \SO[3] group except that, surprisingly, the latter does not admit the half-integer irreps; essentially, \SU[2] has double the irreps because it is the double cover of \SO[3].
Overall, the irreps of \su[2] and \su[3] are quite significant in physics, with direct applications to classical and quantum mechanics, and, moreover, they will also serve as the building blocks for the representations of the Lorentz and Poincaré groups in the next section.

% \section{Particles are irreps of the Poincaré group}
% \label{sec:01_symmetries_poincare}

% The Poincaré group comprises all the physical symmetries of ``flat'' spacetime (i.e, without gravity), i.e. all the transformations which leave the laws of physics invariant.
% These include Lorentz transformations (boosts and rotations) and spacetime translations.

% Particles can be defined as a ``set of states which mix only among themselves under Poincaré transformations'' (Schwartz~\cite{Schwartz:2014sze} Ch. 8.1), leaving attributes like their mass and spin invariant.
% Elementary particles are those for which there is no smaller subset of states that also have this property.
% Thus, they correspond exactly to irreducible representations of the Poincaré group!
% That the physical and seemingly nebulous concept of a particle can be so precisely defined and characterized by a mathematical analysis of the symmetries of spacetime is one of the most beautiful results of fundamental physics.

% In this section, we describe the irreps of the Poincaré group, starting first with the Lorentz group alone.

% % \subsection{Representations of the Lorentz group}

% \subsubsection{The (proper, orthochronous) Lorentz group}

% We know from special relativity that ``flat'' spacetime (i.e., without gravity) is described by 4D Minkowski space $\mathbb R^{1, 3}$.
% This is a real vector space equipped with the metric $\eta_{\mu\nu} = \mathrm{diag}(1, -1, -1, -1)$, which defines distances, or inner products $\langle \cdot, \cdot \rangle$, between 4-vectors $x_\mu = (x_0, x_1, x_2, x_3)$ as:
% \begin{equation}
% 	\label{eq:01_minkowski_metric}
% 	\langle x, y \rangle \equiv x_\mu y^\mu \equiv \eta_{\mu\nu}x^\mu y^\nu = x_0 y_0 - x_1 y_1 - x_2 y_2 - x_3 y_3.
% \end{equation}

% \begin{definition}
% \label{def:01_lorentz_group}
% % Lorentz transformations are those which preserve the magnitude of 4-vectors in Minkowski space $\mathbb R^{1, 3}$.
% The \textit{Lorentz group} is the group of all matrices $M$ orthogonal under the Minkowski metric $M^T\eta M = \eta$, and is called \OO[1, 3].
% This is the analog in flat spacetime to distance-preserving transformations in Euclidean space (e.g., \OO[3]).
% \end{definition}

% \begin{definition}
% \label{def:01_proper_orthochronous}
% The \textit{proper, orthochronous} Lorentz group $\mathrm{SO}^+(1, 3)$ is the subgroup of \OO[1, 3] matrices continuously connected to the identity.
% Physically, these are the transformations that preserve the orientation of space and direction of time, and are typically what we refer to as \textit{Lorentz transformations}.
% The two transformations of \OO[1, 3] not included in $\mathrm{SO}^+(1, 3)$ are parity $P = \diag(1, -1, -1, -1)$ and time reversal $T = \diag(-1, 1, 1, 1)$ (shown in the 4-vector representation), which flip the sign of spatial and temporal components of 4-vectors, respectively.
% Surprisingly, these are not symmetries of nature ---
% they are violated by the weak interaction!
% % \TODO{We discuss this more in ...}
% Generally, in this chapter, when we talk about the Lorentz group or Lorentz invariance, we are referring only to the proper, orthochronous Lorentz group.
% \end{definition}

% \subsubsection{Generators of the Lorentz group}

% Lorentz transformations $\Lambda$ are generated by six antisymmetric matrices, three for boosts ($K_i$) and three for rotations ($J_i$).
% In the $4$-vector representation, these are:
% \begin{equation} 
% 	\begin{split}
% 	\label{eq:01_lorentz_generators}
% 	K_x &= -i\begin{pmatrix}
% 		\setlength{\arraycolsep}{10pt}
% 		0 & 1 & 0 & 0 \\
% 		1 & 0 & 0 & 0 \\
% 		0 & 0 & 0 & 0 \\
% 		0 & 0 & 0 & 0
% 	\end{pmatrix}, \ 
% 	K_y = -i\begin{pmatrix}
% 		0 & 0 & 1 & 0 \\
% 		0 & 0 & 0 & 0 \\
% 		1 & 0 & 0 & 0 \\
% 		0 & 0 & 0 & 0
% 	\end{pmatrix}, \ 
% 	K_z = -i\begin{pmatrix}
% 		0 & 0 & 0 & 1 \\
% 		0 & 0 & 0 & 0 \\
% 		0 & 0 & 0 & 0 \\
% 		1 & 0 & 0 & 0
% 	\end{pmatrix}, \\[1em]
% 	\setlength{\arraycolsep}{5pt}
% 	J_x &= i\begin{pmatrix}
% 		0 & 0 & 0 & 0 \\
% 		0 & 0 & 0 & 0 \\
% 		0 & 0 & 0 & -1 \\
% 		0 & 0 & 1 & 0
% 	\end{pmatrix}, \ 
% 	J_y = i\begin{pmatrix}
% 		0 & 0 & 0 & 0 \\
% 		0 & 0 & 0 & 1 \\
% 		0 & 0 & 0 & 0 \\
% 		0 & -1 & 0 & 0
% 	\end{pmatrix}, \ 
% 	J_z = i\begin{pmatrix}
% 		0 & 0 & 0 & 0 \\
% 		0 & 0 & -1 & 0 \\
% 		0 & 1 & 0 & 0 \\
% 		0 & 0 & 0 & 0
% 	\end{pmatrix}.
% \end{split} 
% \end{equation}
% Lorentz transformations can thus be represented as
% \begin{equation}
% 	\label{eq:01_lorentz_generators_exponential}
% \Lambda(\vec{\theta}, \vec{\beta}) = e^{i(\theta_i J_i + \beta_i K_i)},
% \end{equation}
% where $\vec{\theta}$ and $\vec{\beta}$ are the rotation and boost parameters, respectively,
% % Note that these only represent the transformations of the Lorentz group continuously connected to the identity (by definition of the Lie algebra, Definition~\ref{def:01_lie_algebra}).

% An important property of the Lorentz group is that it is not compact.
% This is related to the fact that the generators for boosts $K_i$ in the representation above are not Hermitian, which means the corresponding group elements $e^{i\beta_i K_i}$ are not unitary.
% In fact, there are no finite-dimensional unitary representations of the Lorentz group~\cite{Wigner:1939cj}.
% Unitarity of operators is an important condition for the invariance of physical properties under transformations in QM, and the consequences of this for the SM will be discussed in Chapter~\ref{sec:01_qft_spinors}.

% \subsubsection{Lie algebra of the Lorentz group}

% From Eq.~\ref{eq:01_lorentz_generators}, we can derive the commutation relations of the generators and, hence, the Lie algebra:
% \begin{equation}
% 	\begin{split}
% 		\label{eq:01_lorentz_algebra}
% 		[K_i, K_j] &= -i \epsilon_{ijk} J_k, \\
% 		 [J_i, J_j] &= i \epsilon_{ijk} J_k, \\
% 		[J_i, K_j] &= i \epsilon_{ijk} K_k.
% 	\end{split}
% \end{equation}
% Moreover, if we define the operators
% \begin{equation}
% 	\label{eq:01_lorentz_su2_operators}
% 	J^+_i = \frac{1}{2}(J_i + iK_i), \quad J^-_i = \frac{1}{2}(J_i - iK_i),
% \end{equation}
% we find that \so[1, 3] contains two mutually commuting \su[2] subalgebras:
% \begin{equation}
% 	\begin{split}
% 		\label{eq:01_lorentz_su2_subalgebras}
% 		[J^+_i, J^+_j] &= i \epsilon_{ijk} J^+_k, \\
% 		[J^-_i, J^-_j] &= i \epsilon_{ijk} J^-_k, \\
% 		[J^+_i, J^-_j] &= 0.
% 	\end{split}
% \end{equation}
% This implies the irreps of \so[1, 3] are simply two copies of the irreps of \su[2] from Section~\ref{sec:01_symmetries_so3}, indexed as $(j_1, j_2)$ with $j_1, j_2 \in \mathbb Z^{\geq0}/2$ and dimension $(2j_1 + 1)(2j_2 + 1)$.

% With this, we can easily obtain the generators $J_i$, $K_i$ for the smallest few irreps:
% % \begin{itemize}
% % 	\item $(0, 0)$: \quad $J^+_i = J^-_i = 0 \Rightarrow J_i = K_i = 0$.
% % 	\item $(1/2, 0)$: \quad $J^+_i = \frac{1}{2} \sigma_i$, $J^-_i = 0 \Rightarrow J_i = \frac{1}{2} \sigma_i$, $K_i = -\frac{i}{2} \sigma_i$.
% % \end{itemize}
% \begin{align}
% 	\label{eq:01_lorentz_irreps_00}
% 	(0, 0)\text{:}& \qquad J^+_i = J^-_i = 0 \! &\Rightarrow& \qquad J_i = K_i = 0. \\
% 	\label{eq:01_lorentz_irreps_weyl_left}
% 	(\nicefrac{1}{2}, 0)\text{:}& \qquad J^+_i = \frac{1}{2} \sigma_i,\, J^-_i = 0 \! &\Rightarrow& \qquad J_i = \frac{1}{2} \sigma_i, K_i = -\frac{i}{2} \sigma_i. \\
% 	\label{eq:01_lorentz_irreps_weyl_right}
% 	(0, \nicefrac{1}{2})\text{:}& \qquad J^+_i = 0,\, J^-_i = \frac{1}{2} \sigma_i \! &\Rightarrow& \qquad J_i = \frac{1}{2} \sigma_i, K_i = \frac{i}{2} \sigma_i. \\[2mm]
% 	% (\nicefrac{1}{2}, \nicefrac{1}{2})\text{:}& 
% 	& \qquad\qquad\qquad\qquad\qquad \vdots \notag
% 	\vspace{-10mm}
% \end{align}
% % \vspace{-2mm}
% % As you can see, the $(\nicefrac{1}{2}, 0)$ and $(0, \nicefrac{1}{2})$ irreps are quite similar, and are called the \textit{left-} and \textit{right-handed Weyl spinor} representations, where the handedness refers to the direction of a particle's spin angular momentum relative to its momentum.
% % are the left- and right-handed Weyl spinors, respectively, of the Lorentz group.
% The $(\nicefrac{1}{2}, \nicefrac{1}{2})$ irrep is actually our familiar $4$-vector representation, but it is more involved to recover the generators in the same form as Eq.~\ref{eq:01_lorentz_generators}.\footnote{See e.g. Ref.~\cite{439080}.}

% % For example, $(0, 0)$ is the trivial representation, with $J^+_i = J^-_i = 0 \Rightarrow J_i = K_i = 0$.
% % The $(1/2, 0)$ representation means $J^+_i = \frac{1}{2} \sigma_i$, $J^-_i = 0 \Rightarrow J_i = \frac{1}{2} \sigma_i$, $K_i = -\frac{i}{2} \sigma_i$, while in $(0, 1/2)$, $J^+_i = 0$, $J^-_i = \frac{1}{2} \sigma_i \Rightarrow J_i = \frac{1}{2} \sigma_i$, $K_i = \frac{1}{2} \sigma_i$.

% \subsubsection{Representations of the Lorentz group}

% It turns out the above four irreps of the Lorentz group are all we need for the SM.
% Their nomenclature and corresponding elementary particle fields are listed in Table~\ref{tab:01_lorentz_representations}.
% Notably, fermions are classified as those with half-integer total spin $j = j_1 + j_2$, and bosons with integer $j$.
% Their radically different behavior is a consequence of the Spin-Statistics theorem~\cite{PhysRev.110.1450} (a notoriously difficult theorem to prove~\cite{FeynmanVol3}), which states that half-integer spin particles obey Fermi-Dirac statistics and integer spin particles Bose-Einstein statistics.

% All known fermionic particle fields live in the $(\nicefrac{1}{2}, 0) \oplus (0, \nicefrac{1}{2})$, or \textit{Dirac spinor}, representation.
% The $(\nicefrac{1}{2}, 0)$ and $(0, \nicefrac{1}{2})$ representations are called the \textit{left-} and \textit{right-handed Weyl spinors} respectively, where the handedness refers to the direction of their spin angular momentum relative to their momentum.
% Physically, this means there is a left-handed and right-handed copy of each fermion, and they have to be packaged together in a Dirac spinor to have masses without violating parity, as we discuss in Chapter~\ref{sec:01_qft_spinors}.
% % \footnote{More fundamentally, this is necessary to resolve gauge anomalies that arise from the quantization of QED and QCD (see Tong SM~\cite{TongSM} Chapter 4.1).}
% % The Dirac spinor thus comprises both a left-handed and right-handed component, which is necessary for QED- and QCD-\textit{gauge-invariant} fermion masses.\footnote{More fundamentally, this is necessary to resolve gauge anomalies that arise from the quantization of QED and QCD (see Tong SM~\cite{TongSM} Chapter 4.1).}
% We will also see that left- and right-handed representations can be equivalently thought of as particles and antiparticles.

% The $(\nicefrac{1}{2}, 0) \oplus (0, \nicefrac{1}{2})$ representation technically also includes real \textit{Majorana spinors} as a subspace, which can represent neutral fermions.
% The only candidate for these in the SM are right-handed neutrinos and, in fact, the existence of such Majorana neutrinos could potentially explain the curiously small \textit{left-handed} neutrino masses through a process called the \textit{seesaw mechanism}~\cite{Foot:1988aq, Schechter:1980gr}.
% To date, however, no experimental evidence for these, such as neutrinoless double beta decay~\cite{Rodejohann:2011mu} or same-sign charged dilepton decays~\cite{CMS:2018jxx}, has been observed.

% On another technical note, the Lorentz group, similar to \SO[3], does not itself admit half-integer, fermionic representations.
% Thus, the true spacetime symmetry group is actually the double cover of \SO[1, 3], $\mathrm{Spin}(1,3)$!
% Indeed, there are many subtleties to the Lorentz group, some of which will be revisited in the context of Lorentz-group equivariant neural networks in Chapter~\ref{sec:06_lgae}.
% To conclude, however, it is worth emphasizing again the remarkable physical insight these seemingly abstract group-theoretic concepts deliver.
% We are able to classify a fundamental dichotomy of particle physics --- bosons versus fermions, and their completely different behavior --- simply by their representation under the Lorentz (or, rather, the $\mathrm{Spin}(1,3)$) group!

% \begin{table}[ht!]
% 	\centering
% 	\renewcommand{\arraystretch}{1.5}
% 	% \setlength{\tabcolsep}{8pt}
% 	% wrap table to textwidth
% 	% \begin{wraptable}{r}{0.5\textwidth}
% 	\resizebox{\textwidth}{!}{
% 	\begin{tabular}{c|c|c}
% 		% \toprule
% 		\textbf{Representation $(j_1, j_2)$} & \textbf{Name} & \textbf{Elementary Fields} \\
% 		\hline
% 		$(0, 0)$ & Scalar & Higgs boson \\
% 		$(\nicefrac{1}{2}, 0)$ & Left-handed Weyl spinor & --- \\
% 		$(0, \nicefrac{1}{2})$ & Right-handed Weyl spinor & --- \\
% 		$(\nicefrac{1}{2}, 0) \oplus (0, \nicefrac{1}{2})$ & Dirac spinor & All fermions \\
% 		$(\nicefrac{1}{2}, \nicefrac{1}{2})$ & Vector & $g, \gamma, W^\pm,$ and $Z$ gauge bosons \\
% 		% \bottomrule
% 	\end{tabular}
% 	}
% 	\vspace{1em}
% 	\caption{Representations of the Lorentz group and their associated particle fields in the SM.}
% 	\label{tab:01_lorentz_representations}
% \end{table}

% % \subsection{Representations of the Poincaré group}

% \subsubsection{Lie algebra of the Poincaré group}

% The Poincaré group is Lorentz transformations plus spacetime translations.
% Just as angular momentum generates rotations, translations are generated by the momentum operator $P_\mu$.
% $P_\mu$ and the Lorentz generators $J_i$ and $K_i$ together comprise the generators of the Poincaré group, and its algebra is thus the Lorentz algebra (Eq.~\ref{eq:01_lorentz_algebra}) plus the commutation relations with the $P_\mu$s:\footnote{See Appendix~\ref{app:01_poincare_algebra} for a derivation.}
% \begin{equation}
% 	\label{eq:01_poincare_algebra}
% 	\begin{split}
% 		[P_\mu, P_\nu] &= 0, \\
% 		[J_i, P_0] &= 0, \\
% 		[J_i, P_j] &= i \epsilon_{ijk}P_k, \\
% 		[K_i, P_0] &= -i P_i, \\
% 		[K_i, P_j] &= i \eta_{ij}P_0.
% 	\end{split}
% \end{equation}
% As is conventional, the Greek indices run over all four spacetime dimensions, while the Latin indices only the three spatial.

% The Poincaré algebra can be expressed more compactly by first combining the Lorentz generators into the antisymmetric tensor $M_{\mu\nu}$:
% \begin{equation}
% \label{eq:01_lorentz_generators_mmunu}
% M_{\mu\nu} = \begin{pmatrix}
% 	0 & K_x & K_y & K_z \\
% 	-K_x & 0 & J_z & -J_y \\
% 	-K_y & -J_z & 0 & J_x \\
% 	-K_z & J_y & -J_x & 0
% \end{pmatrix}
% \quad\Rightarrow \Lambda(\omega) = e^{\frac{i}{2}\omega^{\mu\nu} M_{\mu\nu}},
% \end{equation}
% % M_{0i} = -M_{i0} = K_i, \quad M_{ij} = \epsilon_{ijk} J_k \quad\Rightarrow \Lambda(\omega) = e^{\frac{i}{2}\omega^{\mu\nu} M_{\mu\nu}},
% with $\omega_{\mu\nu}$ another antisymmetric tensor containing the six rotation and boost parameters.
% The algebra can be then written as:
% \begin{equation}
% 	\begin{split}
% 		\label{eq:01_poincare_algebra_mmunu}
% 		[M_{\mu\nu}, M_{\rho\sigma}] &= i(\eta_{\nu\rho}M_{\mu\sigma} - \eta_{\mu\rho}M_{\nu\sigma} - \eta_{\nu\sigma}M_{\mu\rho} + \eta_{\mu\sigma}M_{\nu\rho}), \\
% 		[M_{\mu\nu}, P_\rho] &= i(\eta_{\nu\rho}P_\mu - \eta_{\mu\rho}P_\nu), \\
% 		[P_\mu, P_\nu] &= 0.
% 	\end{split}
% \end{equation}

% \subsubsection{Irreps of the Poincaré group}

% As we saw from Section~\ref{sec:01_symmetries_so3}, we can derive the irreps of an algebra using its Casimir invariants (Definition~\ref{def:01_casimir}).
% Each set of their eigenvalues uniquely labels an irrep, while each basis state within the irreps is indexed by eigenvalues of a maximal set of commuting operators (e.g., $\{J^2, J_z\} \rightarrow \ket{j, m}$ for \so[3]).
% Note that in the following, we simply provide a sketch of the derivations and point to, for example, Zee GT~\cite{Zee:2016fuk} Chapter VII.2 and Tong SM~\cite{TongSM} Chapter 1.1.2 for more detailed proofs and discussion.

% The Casimirs of the Poincaré algebra are the operators
% \begin{equation}
% 	\label{eq:01_poincare_casimirs}
% 	P^2 = P_\mu P^\mu\ \mathrm{ and }\ W^2 = W_\mu W^\mu,
% \end{equation}
% where 
% \begin{equation}
% 	\label{eq:01_poincare_wmu}
% 	W_\mu = \frac{1}{2}\epsilon_{\mu\nu\rho\sigma}M^{\nu\rho}P^\sigma
% \end{equation}
% is the \textit{Pauli-Lubanski vector}, the relativistic analog of the angular momentum operator $J_i$.
% Furthermore, $P_\mu$ commutes with both, and we can label its eigenstates as $\ket{p}$,
% \begin{equation}
% 	\label{eq:01_poincare_pmu_eigenstates}
% 	P_\mu\ket{p} = p_\mu\ket{p}, 
% \end{equation}
% which represent single-particle states with 4-momentum $p_\mu$.
% % Note that we have an infinite number of states and, therefore, necessarily infinite-dimensional representations of the Poincaré group because it is not compact.
% These are therefore eigenstates of $P^2$ as well, with eigenvalues $m^2$, the squared mass of the particle:
% \begin{equation}
% 	\label{eq:01_poincare_p2}
% 	P^2\ket{p} = p_\mu p^\mu\ket{p} = m^2 \ket{p}.
% \end{equation}
% Thus, we see that the mass of a particle, $m$, is one label of the irreps, with states therein indexed by $p_\mu$.
% The other label, the particle spin $j$, is based on the eigenvalue of $W^2$:
% \begin{equation}
% 	\label{eq:01_poincare_w2}
% 	W^2\ket{p, j} \propto j(j+1)\ket{p, j}.
% \end{equation}
% The easiest way to see this is to, for a given $m > 0$, pick a single eigenstate $\ket{p^*}$.
% The simplest is the rest frame $p^*_\mu = (m, 0, 0, 0)$.
% The subgroup of Poincaré transformations which leave $\ket{p^*}$ invariant is called its \textit{little group}.
% In this case, it comprises all 3D rotations --- i.e., \SO[3].
% Indeed, if we look at the Pauli-Lubanski vector acting on $\ket{p^*}$,
% \begin{equation}
% 	\label{eq:01_poincare_w2_eigenstates}
% 	W_\mu\ket{p^*} = \frac{1}{2}\epsilon_{\mu\nu\rho\sigma}M^{\nu\rho}p^{*\sigma}\ket{p^*} \quad\Rightarrow W_0 = 0, W_i = -m J_i,
% \end{equation}
% we simply recover the generators $J_i$ of \so[3].\footnote{This also motivates why $W_\mu$ can be thought of as relativistic angular momentum.}
% Therefore,
% \begin{equation}
% 	\label{eq:01_poincare_w2_eigenvalues}
% 	W^2\ket{p^*, j} = m^2 J^2 \ket{p^*, j} = m^2j(j+1)\ket{p^*, j},
% \end{equation}
% using the eigenvalues of $J^2$ from Eq.~\ref{eq:01_so3_su2_casimir_eigenvalue}.
% Although we chose here to look at a specific state $\ket{p^*}$, this can be shown to hold for all states $\ket{p}$ in the irrep.\footnote{By choosing an eigenstate $\ket{p^*, j}$ of $P^2$ and looking for transformations which leave it invariant, we ``induced'' a subgroup, \SO[3], and used its representation theory to derive the irreps of the Poincaré group. Such a representation is hence called an \textit{induced representation}.}
% Thus, we see that irreps of the Poincaré group, and, hence, particles, are characterized by their mass $m$ and spin $j$.

% \subsubsection{Massive versus massless particles}

% Continuing with the massive, $m > 0$, particle case, we know as well from Section~\ref{sec:01_symmetries_so3} that the eigenstates within the $\ket{p, j}$ irreps are further labeled by their spin along a particular axis: $J_z\ket{p, j, m_j} = m_j\ket{p, j, m_j}$, with $m_j \in \{-j, -j+1, \ldots, j-1, j\}$.
% Thus, massive particles $\ket{m, j}$ exist in $2j+1$ spin states $\otimes$ an infinite number of momentum states $\ket{p},\ p_\mu p^\mu = m^2$.

% However, this is not the case for \textit{massless} particles, which have a different little group.
% Recall that we can never boost into the rest frame of a massless particle to define the simple $\ket{p^*}$ we did above.
% Instead, let us consider the next-best state $\ket{p'}$, $p'_\mu = (E, 0, 0, E)$.
% % It is clearly left invariant by rotations in the $x-y$ plane, but also by two translations generated by:
% % \begin{equation}
% % 	\label{eq:01_poincare_massless_translations}
% % 	A = \begin{pmatrix}
% % 		0 & 1 & 0 & 0 \\
% % 		1 & 0 & 0 & -1 \\
% % 		0 & 0 & 0 & 0 \\
% % 		0 & 1 & 0 & 0
% % 	\end{pmatrix}
% % \end{equation}
% Its little group turns out to be $\mathrm{E}(2)$, the Euclidean group in 2D, whose representations and implications for massless particles are considerably more involved.
% However, the upshot is that it as well has irreps characterized by spin $j$ (and mass $m = 0$), but with only two \textit{helicity} eigenstates therein.

% As it turns out, physically, the mass of particles is based on the strength of their interactions (or lack thereof) with the Higgs boson, the particle at the center of this dissertation.
% We will discuss the mechanisms for these and all fundamental interactions in the next chapter, and see that symmetries are crucial in understanding them.

% % \section{Summary}

% % In this section, we introduced group theory, the mathematical framework for describing symmetries.
% % We described properties of continuous symmetry groups, their generators 

% % and their Lie algebras, defining 

% % and its applications to physics, emphasizing continuous groups and their Lie algebras.
